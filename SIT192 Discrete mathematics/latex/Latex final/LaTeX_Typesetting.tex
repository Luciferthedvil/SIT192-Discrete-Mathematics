\documentclass[addpoints,answers,12pt]{exam} % exam class with 12 point type
\usepackage[T1]{fontenc}                     % replace default font encoding (OT1)
\usepackage{tgschola}                        % font used in the Book of Proof
\usepackage{amsmath,amsthm,amssymb,amsfonts} % packages for mathematical typesetting
\usepackage{pdfpages}                        % include pdf pages with \includepdf{dir/of/page.pdf}
\usepackage[makeroom]{cancel}                % display expressions as cancelled 
\usepackage{minted}
\usepackage{xcolor}
\usepackage[ruled,vlined]{algorithm2e}


\pagestyle{headandfoot} % header and footer style
\firstpageheadrule      % put horizontal line on header of 1st page
\runningheadrule        % put horizontal line on header after 1st page
\firstpageheader{\student}{\subject \mbox{ ---} \assignment}{\dt}
\runningheader{\student}{\subject \mbox{ ---} \assignment}{\dt}
\firstpagefooter{}{}{} % put nothing on footer of 1st page
\runningfooter{}{}{}   % put nothing on footer after 1st page

% set some set symbols
\newcommand{\N}{\mathbb N}
\newcommand{\Z}{\mathbb Z}
\newcommand{\Q}{\mathbb Q}
\newcommand{\R}{\mathbb R}
\newcommand{\C}{\mathbb C}

% redefine tombstone symbol used at the of proofs
\renewcommand{\qedsymbol}{$\blacksquare$}

\newcommand{\student}{Nirosh Ravindran}
\newcommand{\subject}{BSCP|CS|63|134}
\newcommand{\assignment}{LaTeX Typesetting}
\newcommand{\dt}{\today}



\title{LaTeX Typesetting for Discrete Mathematics SIT192}
\author{Nirosh Ravindran BSCP|CS|63|134}

\begin{document}
	\maketitle
	\begin{questions}
		
		\question{Prove that $ \sum_{i=1}^{n} i = \frac{n(n+1)}{2} $ for $ n \ge 0 $, by induction.}
		\begin{solution}
			\begin{proof}\hfil
				
				\textbf{Base case.} When $  n = 0 $, LHS = $ 0 $ and RHS = $ \frac{0\cdot 1}{2} = 0$. Thus RHS = LHS.
				\textbf{Inductive step.} We assumed that $ \sum_{i=1}^{q} i = \frac{q(q+1)}{2} $ is true for an arbitrary fixed integer $ q $ and attempt to prove the validity of the formula for $ (q+1) $. Thus 
				\begin{align*}
					\sum_{i=1}^{q+1} i &=  \underbrace{\sum_{i=1}^{q} i}_{\frac{q(q+1)}{2}} + (q+1) \\
					&= \frac{q(q+1)}{2} + (q+1) \\
					&= \frac{q(q+1) + 2(q+1)}{2} \\
					&= \frac{(q+1)(q+1)}{2} \mbox{.}
				\end{align*} 
				which is precisely the right-side of $ \sum_{i=1}^{n} i = \frac{n(n+1)}{2} $ when $ n = (q+1) $.
				Therefore, by the principle of mathematical induction, $ \sum_{i=1}^{n} i = \frac{n(n+1)}{2} $.
			\end{proof}
		\end{solution}
		\cite{stylianides2007preservice}
	    \vspace{8cm}
	    
	    
	    
	    
		\section{Integrals}
		
		Integral \(\int_{a}^{b} x^2 dx\) inside text.
		
		\medskip
		
		The same integral on display:
		\[
		\int_{a}^{b} x^2 \,dx
		\]
		and multiple integrals:
		\begin{gather*}
			\iint_V \mu(u,v) \,du\,dv
			\\
			\iiint_V \mu(u,v,w) \,du\,dv\,dw
			\\
			\iiiint_V \mu(t,u,v,w) \,dt\,du\,dv\,dw
			\\
			\idotsint_V \mu(u_1,\dots,u_k) \,du_1 \dots du_k
			\\
			\oint_V f(s) \,ds
		\end{gather*}
		\cite{lifanov1996singular}
		
		
		
		
		
		\section{Sums and products}
		
		Sum \(\sum_{n=1}^{\infty} 2^{-n} = 1\) inside text.
		
		The same sum on display:
		\[
		\sum_{n=1}^{\infty} 2^{-n} = 1
		\]
		
		Product \(\prod_{i=a}^{b} f(i)\) inside text.
		
		The same product on display:
		\[
		\prod_{i=a}^{b} f(i)
		\]
		\cite{lipschutz2007theory}
		\vspace{5cm}
		
	
		
		
		
		\definecolor{LightGray}{gray}{0.9}
		Highlighted Code
		
			\begin{minted}
				[
				frame=lines,
				framesep=2mm,
				baselinestretch=1.2,
				bgcolor=LightGray,
				fontsize=\footnotesize,
				linenos
				]
				{python}
				import numpy as np
				
				def incmatrix(genl1,genl2):
				m = len(genl1)
				n = len(genl2)
				M = None #to become the incidence matrix
				VT = np.zeros((n*m,1), int)  #dummy variable
				
				#compute the bitwise xor matrix
				M1 = bitxormatrix(genl1)
				M2 = np.triu(bitxormatrix(genl2),1) 
				
				for i in range(m-1):
				for j in range(i+1, m):
				[r,c] = np.where(M2 == M1[i,j])
				for k in range(len(r)):
				VT[(i)*n + r[k]] = 1;
				VT[(i)*n + c[k]] = 1;
				VT[(j)*n + r[k]] = 1;
				VT[(j)*n + c[k]] = 1;
				
				if M is None:
				M = np.copy(VT)
				else:
				M = np.concatenate((M, VT), 1)
				
				VT = np.zeros((n*m,1), int)
				
				return M
			\end{minted}
	\end{questions}
	
	
	
	

	\vspace{8cm}
	\section{Algorithm Description}
	
	In this section, we describe a simple algorithm for finding the maximum element in an array.
	
	\begin{algorithm}
		\caption{Find Maximum Element in Array}\label{alg:findMax}
		\KwData{Array $A$ of size $n$}
		\KwResult{Maximum element in the array}
		\BlankLine
		\SetAlgoNlRelativeSize{0}
		\SetAlgoNlRelativeSize{-1}
		\SetAlgoNlRelativeSize{-2}
		\For{$i \leftarrow 1$ \KwTo $n$}{
			\If{$A[i] > \text{maxElement}$}{
				$\text{maxElement} \leftarrow A[i]$\;
			}
		}
		\KwRet{$\text{maxElement}$}\;
	\end{algorithm}
	
	The algorithm iterates through each element in the array and updates the maximum element if a larger element is encountered. The final result is the maximum element in the array.
	
	\section{Analysis}
	
	The time complexity of this algorithm is $O(n)$, where $n$ is the size of the array. This is because it iterates through the entire array once.
	
	
	\vspace{10cm}
	\bibliographystyle{plain}
	\bibliography{./bib}
	
\end{document}